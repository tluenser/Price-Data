


\begin{table}
\centering
\scalebox{1}{
  \begin{threeparttable}
    \caption{Database Description (Table 2 of \citet{Cavallo2015})}
     \begin{tabular}{lllllll}
        \toprule
         & {Google Shopping} & {Walmart (abbreviated)} & {Walmart (expanded)} \\
        \midrule
            Queries\tnote{a}       	& 5 	& 18 	& 78 \\
            Locations\tnote{b}	   	& 21 	& 1 	& 1 \\
            Avg. Daily Observations	& 2100  & 678	& 2303  \\        \bottomrule
     \end{tabular}
     
    \begin{tablenotes}
      \small
            \item[a] queries are strings used as search parameters for items (ie. the string typed into a search-bar)
            \item[b] locations are strings used to reference a geographical server from which the string takes place 
    \end{tablenotes}
  \end{threeparttable}
  }
\end{table}
\label{table:scrape1}

\begin{table}
\centering
\scalebox{1}{
  \begin{threeparttable}
    \caption{Database Description (Table 2 of \citet{Cavallo2015}) with New Walmart Data Column}
     \begin{tabular}{lllllll}
        \toprule
         & \( US \) & \( Argentina \) & \( Brazil \) & \( Chile \) & \( Columbia \) & \(Walmart \) \\
        \midrule
Retailers\tnote{a}                       & 4                                 & 1                             & 1                          & 1                         & 1                          & 1  \\
Observations (millions)\tnote{b}         & 28                                & 11                            & 10                         & 10                        & 4                          & 0.03  \\
Products (thousands)\tnote{c}            & 172                               & 28                            & 22                         & 24                        & 9                          & 4  \\
Days\tnote{d}                            & 865                               & 1,041                         & 1,026                      & 1,024                     & 992                        & 18  \\
Initial Date\tnote{e}                    & 03/08                             & 10/07                         & 10/07                      & 10/07                     & 11/07                      & 04/28/21  \\
Final Date\tnote{f}                      & 08/10                             & 08/10                         & 08/10                      & 08/10                     & 08/10                      & current  \\
Categories\tnote{g}                      & 49                                & 74                            & 72                         & 72                        & 59                         & 11  \\
URLs\tnote{h}                            & 16,188                            & 993                           & 322                        & 292                       & 123                        & 6,887  \\
Total missing observations (\%)\tnote{i} & 37                              & 32                            & 26                         & 33                        & 22                         & 0.04  \\ \hline
        \bottomrule
     \end{tabular}
     
    \begin{tablenotes}
      \small
      \item[a] number of retailers from which data is gathered 
      \item[b] total number of observations of price and metadata 
      \item[c] total number of unique products 
      \item[d] total number of days for which data was collected 
      \item[e] initial date of data collection 
      \item[f] final date of data collection 
      \item[g] total number of product categories 
      \item[h] total number of unique product page urls 
      \item[i] total percentage of products for which price data is missing 
    \end{tablenotes}
  \end{threeparttable}
  }
\end{table}
\label{table:cavallo1}


